\documentclass[../main.tex]{subfiles}

\begin{document}

\subsection{Conclusiones}

Tras los resultados obtenidos, se puede confirmar que se han cumplido los objetivos:

\begin{itemize}
    \item Porcentaje de acierto: Se ha conseguido un porcentaje de acierto muy elevado, únicamente teniendo alguna dificultad entre letras cuya representación es muy similar como las letras 'S' y 'T'.
    \item Rendimiento: las pruebas han sido realizadas en un iPhone 12 Pro, el cual tiene un rendimiento superior al exigido por el proyecto. 
    \item Privacidad y consumo de datos: La aplicación realiza todo el procesamiento en el dispositivo, llevando al máximo la privacidad que puede tener el usuario y al mínimo el consumo de datos.
\end{itemize}

Por lo tanto, si se utiliza la aplicación en las condiciones indicadas, los resultados son exitosos.

\subsection{Líneas Futuras}

iOS App para la interpretación de Palabras: Este proyecto se ha establecido como un paso previo a la interpretación de palabras. Esto requiere la interpretación de un gesto dinámico, en vez de una imágen estática. Pero se pueden utilizar las bases sentadas en este proyecto para ello. Es necesario que cada entrada del modelo, sea un conjunto de mediciones de entrada, entre las cuales representan un gesto. Por lo tanto, el modelo de Deep Learning realizará predicciones sobre una secuencia de mediciones, que en su totalidad representan una palabra. 

\end{document}